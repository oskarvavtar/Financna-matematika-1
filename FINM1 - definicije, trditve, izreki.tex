\documentclass[11pt]{article}
\usepackage[utf8]{inputenc}
\usepackage[slovene]{babel}

\usepackage{amsthm}
\usepackage{amsmath, amssymb, amsfonts}
\usepackage{relsize}
\usepackage{marvosym} % \EUR
\usepackage{listings}
\lstset{
	basicstyle=\ttfamily,
	mathescape
}
\usepackage{bbm} % indikatorska

\DeclareMathOperator{\dist}{dist}
\newcommand{\R}{\mathbb{R}}
\newcommand{\N}{\mathbb{N}}
\newcommand{\p}{\mathbb{P}}
\newcommand{\E}{\mathbb{E}}
\newcommand{\M}{\mathcal{M}}
\newcommand{\F}{\mathcal{F}}
\newcommand{\1}{\mathbbm{1}}

\theoremstyle{definition}
\newtheorem{definicija}{Definicija}[section]

\theoremstyle{definition}
\newtheorem{problem}{Problem}[section]

\newtheorem{lema}{Lema}[section]
\newtheorem{vrednost}{Vrednost}
\newtheorem{trditev}{Trditev}[section]
\newtheorem{izrek}{Izrek}[section]


\title{Finančna matematika 1 - definicije, trditve in izreki}
\author{Oskar Vavtar \\
po predavanjih profesorja Janeza Bernika}
\date{2020/21}

\begin{document}
\maketitle
\pagebreak
\tableofcontents
\pagebreak

% #################################################################################################

\section{FINANČNI INŠTRUMENTI}
\vspace{0.5cm}

\begin{definicija}[Obresti]

Naj bo $T$ \textit{časovni horizont} in $R$ \textit{obrestna mera}.

\begin{itemize}

	\item Navadne obresti: 
	\begin{itemize}
		\item Obrestovalni faktor:
		$$A(0, T) ~=~ 1 + RT$$
		\item Diskontni faktor:
		$$D(0, T) ~=~ \frac{1}{1 + RT}$$
	\end{itemize}
	
	\item Diskretno obrestovanje (kapitalizacija)
	$$A(0, 1) ~=~ \left( 1 + \frac{R_k}{k} \right)^k$$
	
	\item Zvezno obrestovanje: 
	$$A(0, 1) ~=~ \lim_{k \rightarrow \infty} \left( 1 + \frac{R_k}{k} \right)^k ~=~ e^{Y \cdot 1}$$	
	
	\item Efektivna obrestna mera:
	\begin{itemize}
		\item Diskretno obrestovanje:
		$$Re ~=~ \left( 1 + \frac{R_k}{k} \right)^k - 1$$
		\item Zvezno obrestovanje:
		$$Re ~=~ e^Y - 1$$
	\end{itemize}
	

\end{itemize}

\end{definicija}
\vspace{0.5cm}

\begin{vrednost}[Kuponska obveznica]

$$P ~=~ \sum_{i=1}^{n-1} C_i \cdot D(0, t_i) + (C_n + N) D(0, t)$$

\end{vrednost}
\vspace{0.5cm}

\begin{vrednost}[Donos do dospetje (Yield to maturity)]

$$P ~=~ \sum_{i=1}^{n-1} \frac{C_i}{(1+y)^i} + \frac{C_n + N}{(1+y)^n}$$

\end{vrednost}
\vspace{0.5cm}

\begin{definicija}[Časovna struktura obrestnih mer]

\begin{align*}
D(0, t) ~&=~ \frac{1}{1 + L(0, t) \cdot t} \\
D(0, t) ~&=~ e^{-Y_t \cdot t}
\end{align*}
Vse inštrumente s fiksnim denarnim tokom lahko vrednostimo po zakonu ene cene.

\end{definicija}
\vspace{0.5cm}

\begin{definicija}[Forward rate agreement]

Dogovor danes (v času $0$) o obrestni meri za obdobje $(t_1, t_2)$, $0 < t_1 < t_2$, z diskontnimi faktorji
$$D(0, t_2) ~=~ D(0, t_1) \cdot D(0, t_1, t_2)$$

\end{definicija}
\vspace{0.5cm}

\begin{definicija}[Terminski posel]

Dogovorimo se v času $0$ o ceni finančnega inštrumenta v času $t>0$ (FRA je terminski posel za nekuponske obveznice). Terminska cena (Forward price):
$$S_0 ~=~ F(t_1) \cdot D(0, t_1) ~~~\Longrightarrow~~~ F(t_1) ~=~ \frac{S_0}{D(0, t_1)}$$
Arbitraža obstaja, ko ta enakost ne velja (\textit{Cash \& carry} ($>$) ter \textit{Inverse cash \& carry} ($<$)).

\end{definicija}
\vspace{0.5cm}

\begin{vrednost}[Convenience yield]

$$F_T ~=~ \frac{S_0 \cdot D_{dollar}(0, T)}{D_{euro}(0, T)} ~=~ S_0 \cdot e^{-\left( Y_{euro}^{(t)} - Y_{dollar}^{(t)} \right) \cdot t}$$

\end{vrednost}
\vspace{0.5cm}

\begin{definicija}[Zamenjave SWAP]

Pri zamenjavi obrestnih mer tipično dolga stran zamenja tokove s fiksno obrestno mero za denarne tokove s spremenljivo obrestno mero. Obrestna mera $R_{\text{SWAP}}$ je določena tako, da je vrednost zamenjave $V_0$ v čaus $t = 0$ enaka $0$. Neto denarni tok za dolgo pozicijo v času $t_i$ je enak 
$$N \cdot (t_i - t_{i-1}) \cdot (R(t_{i-1}, t_i) - R_{\text{SWAP}}).$$

V času $t$ je vrednost zamenjave $V_t$ enaka vsoti vrednosti FRA$_i$, za katere $t_i > t$:
$$V_t ~=~ \sum_{t_i>t} V_t^{\text{FRA}_i} ~=~ N \Delta \sum_{t_i > t} (R(t, t_{i-1}, t_i) - R_{\text{SWAP}}) \cdot D(t, t_i);$$
za $t=0$ more biti $V_0 = 0$, zato je
$$R_\text{SWAP} ~=~ \frac{\sum_{i=1}^n R(0, t_{i-1}, t_i) \cdot D(0, t_i)}{\sum_{i=1}^n D(0, t_i)}.$$

\end{definicija}
\vspace{0.5cm}

\begin{definicija}[Opcija]
~\\
\begin{itemize}
	
	\item Nakupna opcija:
	$$C_T ~=~ (S_T - K)^+ ~=~ \max{\{ S_T - K, 0 \}}$$
	
	\item Prodajna opcija:
	$$P_T ~=~ (K - S_T)^+ ~=~ \max{\{ K - S_T, 0 \}}$$
	
\end{itemize}

Tržne cene ameriških opcij:
\begin{align*}
c_t ~&\geq~ C_t \\
p_t ~&\geq~ P_t
\end{align*}

Časovna vrednost opcije:
\begin{align*}
&c_t - C_t \\
&p_t - C_t
\end{align*}

\end{definicija}
\vspace{0.5cm}

\begin{trditev}

$$\max{\{ S_t - D(t, T) \cdot K, 0\}} ~\leq~ c_t^e ~\leq~ c_t^a ~\leq~ S_t$$

\end{trditev}
\vspace{0.5cm}

\begin{trditev}

Če ni dividend na osnovno premoženje  a $[t, T]$, je
$$p_t^e + S_t ~=~ c_t^e + K \cdot D(t, T)$$
Za ameriške opcije velja:
$$c_t^a + D(t, T) \cdot K ~\leq~ p_t^a + S_t ~\leq~ c_t^a + K$$
po predpostavki, da do $T$ ni dividend. Velja le, če $D(t, T)<1$ in je $D(t, S)$, $0 \leq S \leq T$, padajoča funkcija. 

\end{trditev}
\vspace{0.5cm}

\pagebreak

% #################################################################################################

\section{ENODOBNI MODEL TRGA}
\vspace{0.5cm}

\begin{definicija}[Matrika izplačil]

$K \times N$ matrika, $i$-ti stolpec je vektor izplačil v času $t_1$ za $A_i$
$$M ~=~ \begin{bmatrix}
S_1(\omega_1) & \cdots & S_i(\omega_1) & \cdots & S_N(\omega_1) \\
\vdots & ~ & ~ & ~ & \vdots \\
S_1(\omega_k) & \cdots & S_i(\omega_k) & \cdots & S_N(\omega_k)
\end{bmatrix}$$

\end{definicija}
\vspace{0.5cm}

\begin{definicija}

Vrednostni papir $A_i$:
\begin{itemize}

	\item Netvegan, če je $S_i(\omega_j) = S_i(\omega_l)$ $\forall j,l = 1, \ldots, K$. Izplačilo je konstantno, ne glede na stanje ekonomije v času $t_1$
	\item Tvegan, če obstajata stanji $\omega_j$ in $\omega_l$, da je $S_i(\omega_j) \neq S_i(\omega_l)$

\end{itemize}

\end{definicija}
\vspace{0.5cm}

\begin{vrednost}
\vspace{0.5cm}
~\\
\begin{itemize}

	\item V času $t_0$:
	$$V_0 ~=~ \sum_{i=1}^N c_i \cdot \vartheta_i ~=~ \langle c, \vartheta \rangle$$
	
	\item V času $t_1$ (slučajna spremenljivka):
	$$V_1 ~=~ \sum_{i=1}^N \vartheta \cdot S_i ~=~ M \cdot \vartheta ~=~ \begin{bmatrix}
	V_1(\omega_1) \\
	\vdots \\
	V_1(\omega_k)
	\end{bmatrix}$$

\end{itemize}

\end{vrednost}
\vspace{0.5cm}

\begin{definicija}[Arbitražni portfelj]

Portfelj $\vartheta$ je \textit{arbitražni}, če je $V_0(\vartheta) \leq 0$ in je $V_1(\vartheta)$ pozitiven vektor, ki ima vsaj eno komponentno strogo pozitivno.\\

\noindent Na trgu ne obstaja arbitražna priložnost, če ne obstaja arbitražni portfelj.

\end{definicija}
\vspace{0.5cm}

\begin{definicija}[Pogojna terjatev]

Pogojna terjatev je slučajna spremenljivka $X: \Omega \rightarrow \R$, ki jo lahko podamo z vektorjem $\begin{bmatrix}
X(\omega_1) \\
\vdots \\
X(\omega_k)
\end{bmatrix}$ \\

\noindent Če obstaja portfelj $\vartheta$ tak, da je
$$X ~=~ V_0(\vartheta) ~~~\text{oz.}~~~ X(\omega_j) ~=~ V_1(\vartheta)(\omega_j) ~\forall j$$
je $X$ \textit{dosegljiva} pogojna terjatev. V tem primeru je $\vartheta$ \textit{izvedbeni} portfelj.

\end{definicija}
\vspace{0.5cm}

\begin{definicija}[Popoln model trga]

Trg, ki zadošča našim predpostavkam (ni trenj, vsi enako informirani,\ldots) in v katerem ne obstaja arbitražni portfelj, imenujemo \textit{popoln model trga}. 

\end{definicija}
\vspace{0.5cm}

\begin{definicija}[Zakon ene cene]

Na našem modela velja \textit{zakon ene cene}, če za dva izvedbena portfelja $\vartheta_1, \vartheta_2$ za pogojno terjatev $X$ nujno velja
$$V_0(\vartheta_1) ~=~ V_0(\vartheta_2).$$

\end{definicija}
\vspace{0.5cm}

\begin{trditev}

Trg je \textit{popoln}. $\Longrightarrow$ Velja \textit{zakon ene cene}.

\end{trditev}
\vspace{0.5cm}

\begin{definicija}

Trg je \textit{poln}, če je vsaka pogojna terjatev dosegljiva.

\end{definicija}
\vspace{0.5cm}

\begin{definicija}[Cenovni funkcional]

Denimo, da na našem modelu trga velja \textit{zakon ene cene}. potem lahko definiramo linearni funkcional \hbox{$\pi_0: \M \rightarrow \R$}, s predpisom 
$$\pi_o(x) ~=~ V_0(\vartheta),$$
kjer je $\vartheta$ izvedbeni portfelj za $X$.

\end{definicija}
\vspace{0.5cm}

\pagebreak

\begin{trditev}

Za dani model trga (vse predpostavke razen popolnosti) so ekvivalentne naslednje trditve:
\begin{itemize}

	\item Ne obstaja arbitražni portfelj (popoln trg).
	
	\item Za vsak portfelj $\vartheta$ tak, da je $V_1(\vartheta) \geq 0$ in $V_i(\vartheta) \neq 0$ je $V_0(\vartheta) > 0$.
	
	\item Ne obstaja pozitivna dosegljiva pogojna terjatev $X$ z izvedbenim portfeljem $\vartheta$, da je $V_0(\vartheta) = 0$.
	
	\item Cenovni funkcional $\Pi$ je krepko pozitiven.

\end{itemize}

\end{trditev}
\vspace{0.5cm}

\begin{definicija}[Numerar]

\textit{Numerar} je tak vrednostni papir $A_i$, da je $S_i(\omega_j) > 0$, $\forall j = 1,\ldots, K$. \\

\noindent Definiramo:
\begin{itemize}
	
	\item Relativne cene:
	$$\tilde{C}_k ~=~ \frac{C_k}{C_i}$$
	
	\item Relativne donose:
	$$\tilde{S}_k(\omega_j) ~=~ \frac{S_k(\omega_j)}{S_i(\omega_j)}$$
	
\end{itemize} 

\noindent Vrednost portfelja:
\begin{align*}
\tilde{V}_0(\vartheta) ~&=~ \langle \tilde{C}, \vartheta \rangle \\
\tilde{V}_1(\vartheta) ~&=~ \langle \tilde{S}, \vartheta \rangle
\end{align*} \\

\noindent $(\tilde{V}_0(\vartheta), \tilde{V}_1(\vartheta))$ je \textit{diskontirani} vrednostni proces za $\vartheta$ glede na $A_i$.


\end{definicija}
\vspace{0.5cm}

\begin{definicija}[Martingal]

Diskontirani vrednostni proces $(\tilde{V}_0(\vartheta), \tilde{V}_1(\vartheta))$ je \textit{martingal} glede na neko verjetnost $Q$ na $\Omega$, če velja
$$\E_Q(\tilde{V}_1(\vartheta)) ~=~ \tilde{V}_0(\vartheta).$$

\end{definicija}
\vspace{0.5cm}

\begin{izrek}[1.\,osnovni izrek vrednotenja premoženja]

Naj bo dan enodobni model trga (z vsemi predpostavkami razen neobstoja arbitraže). $A_1$ naj bo \textit{numerar}. Potem je trg brez arbitraže (popoln) natanko tedaj, ko na $\Omega$ obstaja ekvivalentna verjetnost $Q$ taka, da je diskontiran osnovni proces za vsak $A_i$, $1 \leq i \leq N$, \textit{martingal} glede na $Q$.
$$\tilde{C}_i ~=~ \E_Q(\tilde{S}_i), ~i = 1,\ldots, N ~~~\Longrightarrow~~~ \tilde{V}_0(\vartheta) ~=~ \tilde{V}_1(\vartheta) ~\forall \vartheta.$$

\end{izrek}
\vspace{0.5cm}

\begin{izrek}[2.\,osnovni izrek vrednotenja premoženja]

Če je trg \textit{popoln}, torej za dan $A_1$ obstaja ekvivalentna martingalska verjetnost $Q$, je ta enolična natanko tedaj, ko je trg \textit{poln}.

\end{izrek}
\vspace{0.5cm}

\pagebreak

% #################################################################################################

\section{VEČOBDOBNI MODEL TRGA}
\vspace{0.5cm}

\begin{definicija}

Če je $\F_0$ generirana z atomi (najmanjša neprazna množica, ki je še v $\F_t$) $A_1^t,\ldots, A_{k_t}^t$, je $\F_t$ \textit{merljiva} natanko tedaj, ko je konstantna na vsakem od atomov $A_j^t$, $1 \leq j \leq k_t$. ($\F_t$ merljiva $\iff$ poznana v času $t$)

\end{definicija}
\vspace{0.5cm}

\begin{definicija}

Zaporedje $V = (V_0, V_1,\ldots, V_T)$ je \textit{prilagojen proces} filtraciji $(\F_0, \F_1,\ldots, \F_T)$, če je $V_t$ \textit{merljiva} glede na $\F_t$, za $0 \leq t \leq T$. V posebnem primeru, ker je $\F_0 = \{\emptyset, \Omega\}$, je potem $V_0$ konstantna.

\end{definicija}
\vspace{0.5cm}

\begin{definicija}

Proces $(V_0,\ldots, V_T)$ je \textit{napovedljiv}, če je $V_t$ merljiva glede na $\F_{t-1}$, $t=1,\ldots, T$, $V_0$ pa konstantna. To pomeni: če je proces \textit{napovedljiv}, vem v času $t-1$, kaj bo v času $t$.

\end{definicija}
\vspace{0.5cm}

\begin{trditev}

Za vsak dinamični portfelj imamo:
\begin{itemize}

	\item Likvidacijsko vrednost v času $t$:
	$$V_t^L(\vartheta) ~=~ \sum_{j=1}^N \vartheta_t^j \cdot S_t^j ~~~(V_t^L)_{t=0}^S ~\text{je prilagojen}$$
	
	\item Nabavno vrednost v času $t$:
	$$V_t^A(\vartheta) ~=~ \sum_{j=1}^N \vartheta_{t+1}^j \cdot S_t^j ~~~(V_t^A)_{t=1}^{N-1} ~\text{je prilagojen}$$

\end{itemize}

\end{trditev}
\vspace{0.5cm}

\begin{definicija}[Strategija samofinanciranje]

\textit{Strategija samofinanciranja} je taka trgovalna strategija $(\vartheta_1, \ldots, \vartheta_S)$, $S \leq T$, za katero je 
$$V_t^L(\vartheta) ~=~ V_t^A(\vartheta), ~~~ 1 \leq t \leq S-1.$$
To pomeni, da v vmesnem času od $0$ do $S$ ni vmesnih denarnih tokov, ne pozitivnih, ne negativnih. Tedaj govorimo kar o $(V_t(\vartheta))_{t=0}^T$ kot vrednostnem procesu strategije samofinanciranja.

\end{definicija}
\vspace{0.5cm}

\begin{definicija}

Atom za $\F_k$ je najmanjša podmnožica $\Omega$, za katero lahko v času $k$ povemo, ali se je zgodila ali ne.

\end{definicija}
\vspace{0.5cm}

\begin{definicija}

Če imamo dano filtracijo $\F_0 \subseteq \F_1 \subseteq \ldots \subseteq \F_T$ je $(X_i)_{i=0}^T$ \textit{prilagojen} tej filtraciji, če je $X_i$ $\F_i$-merljiva, $\forall i \in \{0,\ldots, T\}$ (torej je $X_i$ konstantna na atomih, ki generirajo $\F_i$). $(S_i)_{i=0}^T$ je \textit{prilagojen} $\forall j = 1,\ldots, K$.

\end{definicija}
\vspace{0.5cm}

\begin{definicija}

$(Y_i)_{i=0}^T$ je \textit{predvidljiv}, če je $Y_i$ $\F_i$-merljiv za $\forall i > 1$ in $Y_0$ konstantna.

\end{definicija}
\vspace{0.5cm}

\begin{trditev}
~\\
\textit{Zakon ene cene} velja za dospelost $T$ $\iff$ velja za vse dospelosti $1 \leq S \leq T$.

\end{trditev}
\vspace{0.5cm}

\begin{definicija}[Arbitražni portfelj]

Strategija samofinanciranja $\vartheta$ z dospelostjo $S$ je \textit{arbitražna strategija}, če 
$$V_0^A(\vartheta) = V_0(\vartheta) \leq 0, ~~~V_i(\vartheta) \geq 0, ~1 \leq i \leq S ~~~\text{in}~~~ V_S(\vartheta) \neq 0.$$

\end{definicija}
\vspace{0.5cm}

\begin{trditev}

Ne obstaja arbitražna strategija z dospelostjo $T$ $\iff$ ne obstaja za dospelost $S$, $1 \leq S \leq T$.

\end{trditev}
\vspace{0.5cm}

\begin{trditev}

Trg je \textit{popoln} $\Longrightarrow$ velja \textit{zakon ene cene}.

\end{trditev}
\vspace{0.5cm}

\begin{izrek}

Na trgu ni arbitraže (z dospelostjo $S$) $\iff$ $\pi_S$ strogo (\hbox{$X_S \in \M_S$}, $X_S \geq 0$ in $X_S \neq 0 ~\Rightarrow~ \pi_S(X_S) > 0$). 

$$\pi_S: \M_S \rightarrow \R, ~~~\pi_S(X_S) ~=~ V_0(\vartheta); ~~~\vartheta ~\text{izvedbena strategija za}~ X_S.$$

\end{izrek}
\vspace{0.5cm}

\begin{trditev}

Trg je \textit{poln} za dospelost $T$ $\iff$ poln je za vse ostale dospelosti.

\end{trditev}
\vspace{0.5cm}

\begin{izrek}

Denimo, da velja \textit{zakon ene cene}. Potem na trgu ni arbitraže $\iff$ cenovni funkcional $\pi_S$ ima krepko pozitivno rešitev $\tilde{\pi}_S$ na $\mathcal{L}_S(\Omega)$ (ni enolična, če trg ni \textit{poln}).

\end{izrek}
\vspace{0.5cm}

\begin{definicija}[Martingal]

Filtriran verjetnostni prostor. Prilagojen proces $(X_i)_{i=1}^T$ je \textit{martingal} glede na $\F_i$ in $P$, če je 
$$X_i ~=~ \E(X_{i+1} \mid \F_i), ~~~i = 0, \ldots, T-1 ~~~(\F_0 \subseteq \ldots \subseteq \F_t).$$ 

\end{definicija}
\vspace{0.5cm}

\begin{trditev}[Stolpna lastnost]

$\F_i \subseteq \F_j$:
$$\E(\E(X_T \mid \F_j) \mid \F_i) ~=~ \E(X_T \mid \F_i)$$

\end{trditev}
\vspace{0.5cm}

\begin{definicija}[Submartingal]

Prilagojen proces $(X_i)_{i=1}^T$ je \textit{podmartingal (submartingal)}, če velja
$$X_i ~\leq~ \E(X_{i+1} \mid \F_i), ~~~i=0, \ldots, T-1.$$

\end{definicija}
\vspace{0.5cm}

\begin{definicija}[Supermartingal]

Prilagojen proces $(X_i)_{i=1}^T$ je \textit{nadmartingal (supermartingal)}, če velja
$$X_i ~\geq~ \E(X_{i+1} \mid \F_i), ~~~i=0, \ldots, T-1.$$

\end{definicija}
\vspace{0.5cm}

\begin{trditev}[Doobova dekompozicija]

Naj bo $(X_n)_{n=0}^T$ prilagojen proces glede na $(\F_n)_{n=0}^T$. Potem obstaja martingal $(M_n)_{n=0}^T$ in predvidljiv proces $(A_n)_{n=0}^T$, taka da je
$$X_n ~=~ M_n + A_n, ~~~m = 0, \ldots, T.$$
Če zahtevamo še $A_0 = 0$, potem je ta dekompozicija enolična.

\end{trditev}
\vspace{0.5cm}

\begin{trditev}

$Q$ je ekvivalentna martingalska verjetnost $\iff$ diskontirani vrednostni procesi samofinancirajočih strategij so martingali.

\end{trditev}
\vspace{0.5cm}

\begin{izrek}[1.\,osnovni izrek vrednotenja premoženja]

Obstoj ekvivalentne martingalske verjetnosti pri vsakem numerarju je ekvivalenten pogoju, da je trg brez arbitraže.

\end{izrek}
\vspace{0.5cm}

\begin{izrek}[2.\,osnovni izrek vrednotenja premoženja]

Če je trg popoln (torej ne obstaja arbitraža), potem je pri izbranem numerarju ekvivalentna martingalska verjetnost ena sama natanko tedaj, ko je trg poln.

\end{izrek}
\vspace{0.5cm}

\pagebreak

% #################################################################################################

\section{BINOMSKI MODEL}
\vspace{0.5cm}

\begin{definicija}[Binomski model]

Imamo $T$ obdobij. Na vsakem koraku imamo dva razvoja: dobrega in slabega; na vsakem koraku je pogojna verjetnost dobrega razvoja $p$, slabega pa $q = 1-p$. 
$$\Omega ~=~ \{(\omega_1,\ldots, \omega_T \mid \omega_i \in \{d, s\}\}$$
$\F_k$ je generirana z $2^k$ atomi. Štejemo, kolikokrat je bil razvoj dober do časa $t$:
\begin{align*}
D_t ~&=~ \sum_{k=1}^t \1_{\{w_k = d\}}, ~~~1 \leq t \leq T \\
D_t ~&\sim~ \text{Bin}(t, p) \\
\1_{\{w_k = d\}} ~&=~ Z_k, ~~~k = 1,\ldots,T \\
Z_k ~&\sim~ \text{Ber}(p)
\end{align*}

\end{definicija}
\vspace{0.5cm}

\begin{definicija}

Ločimo:
\begin{itemize}

	\item \textit{State dependent}: pogojne terjatve, katerih vrednost v času $t$ je odvisna le od $S_t$ (ne pa od dejanskega razvoja).
	
	\item \textit{Path dependent}: pogojne terjatve, ki so v času $t$ odvisne od zgodovine razvoja cenovnega procesa $(S_0, S_1, \ldots, S_t)$.

\end{itemize}

\end{definicija}
\vspace{0.5cm}

\begin{definicija}[Časi ustavljanja]

$(\F)_{t=0}^N$ je filtracija na $(\Omega, \F, \p)$. Naj bo $T: \Omega \rightarrow \{0, 1, \ldots, N\}$ slučajna spremenljivka, za katero velja $\{T = k\} \in \F_k$. Velja: $T, S$ časa ustavljanja $\Longrightarrow$ $T \land S = \min\{T, S\}$, $T \lor S = \min\{T, S\}$ časa ustavljanja. \\

\noindent $\tau: \Omega \rightarrow \{0,\ldots, T\} \cup \{\infty\}$ je \textit{čas ustavljanja}, če je
$$\{\tau = k\} \in \F_k, ~\forall k = 0,\ldots, T ~~~\iff~~~ \{\tau \leq k\} \in \F_k, ~\forall k = 0,\ldots, T$$
$0 < S < t \leq t$:
$$\tau_{S, t} ~=~ \{\tau; ~\tau ~\text{je čas ustavljanja,}~ S \leq \tau(\omega) \leq t ~\forall \omega \in \Omega\}$$
$\sigma$-algebra zgodovine $\tau$:
$$\F_\tau ~=~ \{A \in \F; ~A \cap \{T=k\} \in \F_k, ~\forall k = 0,\ldots, T\}.$$

\end{definicija}
\vspace{0.5cm}

\begin{trditev}

Naj bo $(X_n)_{n=0}^T$ prilagojen proces $(\F_n)_{n=0}^T$, $\tau$ je čas ustavljanja glede na $\F_n$. Ustavljeni proces $(X_{n \land \tau})_{n=0}^T$ je tudi prilagojen, torej je $X_{n \land \tau}$ $\F_n$-merljiva, $n = 0, \ldots, T$.

\end{trditev}
\vspace{0.5cm}

\begin{trditev}

Naj bo $(X_n)_{n=0}^T$ martingal. Potem je $(X_{n \land \tau})_{n=0}^T$ tudi martingal. Enako velja za submartingale in supermartingale.

\end{trditev}
\vspace{0.5cm}

\begin{trditev}

Za dva prilagojena procesa $(X_n)_{n=0}^T$, $(Y_n)_{n=0}^T$ rečemo, da $Y$ dominira $X$, če velja $Y_n(\omega) \geq X_n(\omega)$, $\forall n = 0, \ldots, T$, $\forall \omega \in \Omega$. Če $Y$ dominira $X$, potem $Y_{n \land \tau}$ dominira $X_{n \land \tau}$ $\forall \tau$ (čas ustavljanja).

\end{trditev}
\vspace{0.5cm}

\begin{definicija}[Smellova ovojnica in Bellmanov algoritem]

$(V_0, V_1, \ldots, V_T)$ je prilagojen proces. Včasu $T-1$ se odločamo: se bolj splača nadaljevati ali ustaviti? $\E(V_T \mid \F_{T-1})$ in $V_{T-1}$ sta merljivi glede na $\F_{T-1}$:
$$U_{T-1} ~=~ \max\{V_{T-1}, \E(V_t \mid \F_{T-1}\}.$$

\end{definicija}
\vspace{0.5cm}

\pagebreak

% #################################################################################################

\end{document}